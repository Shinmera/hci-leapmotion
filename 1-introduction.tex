\subsection{Introduction}
The Leap Motion is a gesture-based input device. It uses a set of sensors in order to visually recognise three-dimensional objects, in particular human hands, and represent them digitally. This allows it to interpret various motions and hand gestures as input for computer programs. Particularly, the Leap Motion is advertised as being applicable for a rather wide range of uses, including drawing, spatial navigation, CAD, and video games \citep{introduction-advertisement}.

In this paper we analyse the usability of the Leap Motion as an input device for a particular game and contrast it against a traditional gamepad. While the Leap Motion has shown to have greater accuracy than rivalling products focused on game input like the Microsoft Kinect \citep{introduction-accuracy}, it has different setup requirements and is limited to hand gestures. While motion based game input controllers such as the Nintendo Wii and Microsoft Kinect have reached wide market adoption, and have been shown to have good promise for usability \citep{introduction-wiikinect}, the Leap Motion presents a different setup that limits itself to a much more narrow environment that restricts the user's movements. With the Wii and the Kinect the player can move around freely in their living room, but with the Leap Motion the user needs to constrain their hands to the view area of the device. This, combined with the limitation to gestures, presents unique challenges to the usability for video games. We were particularly interested in how the Leap Motion influences the sense of immersion for the player, as well as the fun-factor of the game.

The usability and accuracy of the Leap Motion has been analysed in other papers and has shown mixed results \citep{introduction-pointing} \citep{introduction-markerless} \citep{introduction-pointing3d}. However, not much research could be found regarding the specific case of video game interaction, especially in a broader context, rather than as a case-study.

%%% Local Variables:
%%% mode: latex
%%% TeX-master: "hci-leapmotion"
%%% TeX-engine: luatex
%%% End:
